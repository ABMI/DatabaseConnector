\documentclass[]{article}
\usepackage{lmodern}
\usepackage{amssymb,amsmath}
\usepackage{ifxetex,ifluatex}
\usepackage{fixltx2e} % provides \textsubscript
\ifnum 0\ifxetex 1\fi\ifluatex 1\fi=0 % if pdftex
  \usepackage[T1]{fontenc}
  \usepackage[utf8]{inputenc}
\else % if luatex or xelatex
  \ifxetex
    \usepackage{mathspec}
  \else
    \usepackage{fontspec}
  \fi
  \defaultfontfeatures{Ligatures=TeX,Scale=MatchLowercase}
\fi
% use upquote if available, for straight quotes in verbatim environments
\IfFileExists{upquote.sty}{\usepackage{upquote}}{}
% use microtype if available
\IfFileExists{microtype.sty}{%
\usepackage{microtype}
\UseMicrotypeSet[protrusion]{basicmath} % disable protrusion for tt fonts
}{}
\usepackage[margin=1in]{geometry}
\usepackage{hyperref}
\hypersetup{unicode=true,
            pdftitle={Using DatabaseConnector},
            pdfauthor={Martijn J. Schuemie},
            pdfborder={0 0 0},
            breaklinks=true}
\urlstyle{same}  % don't use monospace font for urls
\usepackage{color}
\usepackage{fancyvrb}
\newcommand{\VerbBar}{|}
\newcommand{\VERB}{\Verb[commandchars=\\\{\}]}
\DefineVerbatimEnvironment{Highlighting}{Verbatim}{commandchars=\\\{\}}
% Add ',fontsize=\small' for more characters per line
\usepackage{framed}
\definecolor{shadecolor}{RGB}{248,248,248}
\newenvironment{Shaded}{\begin{snugshade}}{\end{snugshade}}
\newcommand{\AlertTok}[1]{\textcolor[rgb]{0.94,0.16,0.16}{#1}}
\newcommand{\AnnotationTok}[1]{\textcolor[rgb]{0.56,0.35,0.01}{\textbf{\textit{#1}}}}
\newcommand{\AttributeTok}[1]{\textcolor[rgb]{0.77,0.63,0.00}{#1}}
\newcommand{\BaseNTok}[1]{\textcolor[rgb]{0.00,0.00,0.81}{#1}}
\newcommand{\BuiltInTok}[1]{#1}
\newcommand{\CharTok}[1]{\textcolor[rgb]{0.31,0.60,0.02}{#1}}
\newcommand{\CommentTok}[1]{\textcolor[rgb]{0.56,0.35,0.01}{\textit{#1}}}
\newcommand{\CommentVarTok}[1]{\textcolor[rgb]{0.56,0.35,0.01}{\textbf{\textit{#1}}}}
\newcommand{\ConstantTok}[1]{\textcolor[rgb]{0.00,0.00,0.00}{#1}}
\newcommand{\ControlFlowTok}[1]{\textcolor[rgb]{0.13,0.29,0.53}{\textbf{#1}}}
\newcommand{\DataTypeTok}[1]{\textcolor[rgb]{0.13,0.29,0.53}{#1}}
\newcommand{\DecValTok}[1]{\textcolor[rgb]{0.00,0.00,0.81}{#1}}
\newcommand{\DocumentationTok}[1]{\textcolor[rgb]{0.56,0.35,0.01}{\textbf{\textit{#1}}}}
\newcommand{\ErrorTok}[1]{\textcolor[rgb]{0.64,0.00,0.00}{\textbf{#1}}}
\newcommand{\ExtensionTok}[1]{#1}
\newcommand{\FloatTok}[1]{\textcolor[rgb]{0.00,0.00,0.81}{#1}}
\newcommand{\FunctionTok}[1]{\textcolor[rgb]{0.00,0.00,0.00}{#1}}
\newcommand{\ImportTok}[1]{#1}
\newcommand{\InformationTok}[1]{\textcolor[rgb]{0.56,0.35,0.01}{\textbf{\textit{#1}}}}
\newcommand{\KeywordTok}[1]{\textcolor[rgb]{0.13,0.29,0.53}{\textbf{#1}}}
\newcommand{\NormalTok}[1]{#1}
\newcommand{\OperatorTok}[1]{\textcolor[rgb]{0.81,0.36,0.00}{\textbf{#1}}}
\newcommand{\OtherTok}[1]{\textcolor[rgb]{0.56,0.35,0.01}{#1}}
\newcommand{\PreprocessorTok}[1]{\textcolor[rgb]{0.56,0.35,0.01}{\textit{#1}}}
\newcommand{\RegionMarkerTok}[1]{#1}
\newcommand{\SpecialCharTok}[1]{\textcolor[rgb]{0.00,0.00,0.00}{#1}}
\newcommand{\SpecialStringTok}[1]{\textcolor[rgb]{0.31,0.60,0.02}{#1}}
\newcommand{\StringTok}[1]{\textcolor[rgb]{0.31,0.60,0.02}{#1}}
\newcommand{\VariableTok}[1]{\textcolor[rgb]{0.00,0.00,0.00}{#1}}
\newcommand{\VerbatimStringTok}[1]{\textcolor[rgb]{0.31,0.60,0.02}{#1}}
\newcommand{\WarningTok}[1]{\textcolor[rgb]{0.56,0.35,0.01}{\textbf{\textit{#1}}}}
\usepackage{graphicx,grffile}
\makeatletter
\def\maxwidth{\ifdim\Gin@nat@width>\linewidth\linewidth\else\Gin@nat@width\fi}
\def\maxheight{\ifdim\Gin@nat@height>\textheight\textheight\else\Gin@nat@height\fi}
\makeatother
% Scale images if necessary, so that they will not overflow the page
% margins by default, and it is still possible to overwrite the defaults
% using explicit options in \includegraphics[width, height, ...]{}
\setkeys{Gin}{width=\maxwidth,height=\maxheight,keepaspectratio}
\IfFileExists{parskip.sty}{%
\usepackage{parskip}
}{% else
\setlength{\parindent}{0pt}
\setlength{\parskip}{6pt plus 2pt minus 1pt}
}
\setlength{\emergencystretch}{3em}  % prevent overfull lines
\providecommand{\tightlist}{%
  \setlength{\itemsep}{0pt}\setlength{\parskip}{0pt}}
\setcounter{secnumdepth}{5}
% Redefines (sub)paragraphs to behave more like sections
\ifx\paragraph\undefined\else
\let\oldparagraph\paragraph
\renewcommand{\paragraph}[1]{\oldparagraph{#1}\mbox{}}
\fi
\ifx\subparagraph\undefined\else
\let\oldsubparagraph\subparagraph
\renewcommand{\subparagraph}[1]{\oldsubparagraph{#1}\mbox{}}
\fi

%%% Use protect on footnotes to avoid problems with footnotes in titles
\let\rmarkdownfootnote\footnote%
\def\footnote{\protect\rmarkdownfootnote}

%%% Change title format to be more compact
\usepackage{titling}

% Create subtitle command for use in maketitle
\providecommand{\subtitle}[1]{
  \posttitle{
    \begin{center}\large#1\end{center}
    }
}

\setlength{\droptitle}{-2em}

  \title{Using DatabaseConnector}
    \pretitle{\vspace{\droptitle}\centering\huge}
  \posttitle{\par}
    \author{Martijn J. Schuemie}
    \preauthor{\centering\large\emph}
  \postauthor{\par}
      \predate{\centering\large\emph}
  \postdate{\par}
    \date{2019-07-16}


\begin{document}
\maketitle

{
\setcounter{tocdepth}{2}
\tableofcontents
}
\hypertarget{introduction}{%
\section{Introduction}\label{introduction}}

DatabaseConnector is an R package for connecting to various database
platforms using Java's JDBC drivers.

Supported database platforms:

\begin{itemize}
\tightlist
\item
  Microsoft SQL Server
\item
  Oracle
\item
  PostgresSql
\item
  Microsoft Parallel Data Warehouse (a.k.a. Analytics Platform System)
\item
  Amazon Redshift
\item
  Apache Impala
\item
  Google BigQuery
\item
  IBM Netezza
\item
  SQLite
\end{itemize}

\hypertarget{obtaining-drivers-for-bigquery-netezza-and-impala}{%
\section{Obtaining drivers for BigQuery, Netezza and
Impala}\label{obtaining-drivers-for-bigquery-netezza-and-impala}}

The package already contains most drivers, but because of licensing
reasons the drivers for BigQuery, Netezza and Impala are not included
but must be obtained by the user. Type

\begin{Shaded}
\begin{Highlighting}[]
\NormalTok{?jdbcDrivers}
\end{Highlighting}
\end{Shaded}

for instructions on how to download these drivers. Once downloaded, you
can use the \texttt{pathToDriver} argument of the \texttt{connect},
\texttt{dbConnect}, and \texttt{createConnectionDetails} functions.

\hypertarget{creating-a-connection}{%
\section{Creating a connection}\label{creating-a-connection}}

To connect to a database a number of details need to be specified, such
as the database platform, the location of the server, the user name, and
password. We can call the \texttt{connect} function and specify these
details directly:

\begin{Shaded}
\begin{Highlighting}[]
\NormalTok{conn <-}\StringTok{ }\KeywordTok{connect}\NormalTok{(}\DataTypeTok{dbms =} \StringTok{"postgresql"}\NormalTok{,}
                \DataTypeTok{server =} \StringTok{"localhost/postgres"}\NormalTok{,}
                \DataTypeTok{user =} \StringTok{"joe"}\NormalTok{,}
                \DataTypeTok{password =} \StringTok{"secret"}\NormalTok{,}
                \DataTypeTok{schema =} \StringTok{"cdm"}\NormalTok{)}
\end{Highlighting}
\end{Shaded}

\begin{verbatim}
#> Connecting using PostgreSQL driver
\end{verbatim}

See \texttt{?connect} for information on which details are required for
each platform. Don't forget to close any connection afterwards:

\begin{Shaded}
\begin{Highlighting}[]
\KeywordTok{disconnect}\NormalTok{(conn)}
\end{Highlighting}
\end{Shaded}

Note that, instead of providing the server name, it is also possible to
provide the JDBC connection string if this is more convenient:

\begin{Shaded}
\begin{Highlighting}[]
\NormalTok{conn <-}\StringTok{ }\KeywordTok{connect}\NormalTok{(}\DataTypeTok{dbms =} \StringTok{"postgresql"}\NormalTok{,}
                \DataTypeTok{connectionString =} \StringTok{"jdbc:postgresql://localhost:5432/postgres"}\NormalTok{,}
                \DataTypeTok{user =} \StringTok{"joe"}\NormalTok{,}
                \DataTypeTok{password =} \StringTok{"secret"}\NormalTok{,}
                \DataTypeTok{schema =} \StringTok{"cdm"}\NormalTok{)}
\end{Highlighting}
\end{Shaded}

\begin{verbatim}
#> Connecting using PostgreSQL driver
\end{verbatim}

Sometimes we may want to first specify the connection details, and defer
connecting until later. This may be convenient for example when the
connection is established inside a function, and the details need to be
passed as an argument. We can use the \texttt{createConnectionDetails}
function for this purpose:

\begin{Shaded}
\begin{Highlighting}[]
\NormalTok{details <-}\StringTok{ }\KeywordTok{createConnectionDetails}\NormalTok{(}\DataTypeTok{dbms =} \StringTok{"postgresql"}\NormalTok{,}
                                   \DataTypeTok{server =} \StringTok{"localhost/postgres"}\NormalTok{,}
                                   \DataTypeTok{user =} \StringTok{"joe"}\NormalTok{,}
                                   \DataTypeTok{password =} \StringTok{"secret"}\NormalTok{,}
                                   \DataTypeTok{schema =} \StringTok{"cdm"}\NormalTok{)}
\NormalTok{conn <-}\StringTok{ }\KeywordTok{connect}\NormalTok{(details)}
\end{Highlighting}
\end{Shaded}

\begin{verbatim}
#> Connecting using PostgreSQL driver
\end{verbatim}

\hypertarget{specifying-the-driver-location-for-bigquery-netezza-and-impala}{%
\subsection{Specifying the driver location for BigQuery, Netezza and
Impala}\label{specifying-the-driver-location-for-bigquery-netezza-and-impala}}

For BigQuery, Netezza and Impala the drivers are not included in the
\texttt{DatabaseConnector} package and need to be downloaded separately,
as noted earlier. Once downloaded, we can point to the folder containing
the jar files using the \texttt{pathToDriver} argument:

\begin{Shaded}
\begin{Highlighting}[]
\NormalTok{details <-}\StringTok{ }\KeywordTok{createConnectionDetails}\NormalTok{(}\DataTypeTok{dbms =} \StringTok{"netezza"}\NormalTok{,}
                                   \DataTypeTok{server =} \StringTok{"myserver.com/mainDb"}\NormalTok{,}
                                   \DataTypeTok{user =} \StringTok{"joe"}\NormalTok{,}
                                   \DataTypeTok{password =} \StringTok{"secret"}\NormalTok{,}
                                   \DataTypeTok{schema =} \StringTok{"cdm"}\NormalTok{,}
                                   \DataTypeTok{pathToDriver =} \StringTok{"c:/temp"}\NormalTok{)}
\NormalTok{conn <-}\StringTok{ }\KeywordTok{connect}\NormalTok{(details)}
\end{Highlighting}
\end{Shaded}

\begin{verbatim}
#> Connecting using Netezza driver
\end{verbatim}

\hypertarget{querying}{%
\section{Querying}\label{querying}}

The main functions for querying database are the \texttt{querySql} and
\texttt{executeSql} functions. The difference between these functions is
that \texttt{querySql} expects data to be returned by the database, and
can handle only one SQL statement at a time. In contrast,
\texttt{executeSql} does not expect data to be returned, and accepts
multiple SQL statements in a single SQL string.

Some examples:

\begin{Shaded}
\begin{Highlighting}[]
\KeywordTok{querySql}\NormalTok{(conn, }\StringTok{"SELECT TOP 3 * FROM person"}\NormalTok{)}
\end{Highlighting}
\end{Shaded}

\begin{verbatim}
#>   PERSON_ID GENDER_CONCEPT_ID YEAR_OF_BIRTH
#> 1         1              8507          1975
#> 2         2              8507          1976
#> 3         3              8507          1977
\end{verbatim}

\begin{Shaded}
\begin{Highlighting}[]
\KeywordTok{executeSql}\NormalTok{(conn, }\StringTok{"TRUNCATE TABLE foo; DROP TABLE foo; CREATE TABLE foo (bar INT);"}\NormalTok{)}
\end{Highlighting}
\end{Shaded}

Both function provide extensive error reporting: When an error is thrown
by the server, the error message and the offending piece of SQL are
written to a text file to allow better debugging. The
\texttt{executeSql} function also by default shows a progress bar,
indicating the percentage of SQL statements that has been executed. If
those attributes are not desired, the package also offers the
\texttt{lowLevelQuerySql} and \texttt{lowLevelExecuteSql} functions.

\hypertarget{querying-using-ffdf-objects}{%
\subsection{Querying using ffdf
objects}\label{querying-using-ffdf-objects}}

Sometimes the data to be fetched from the database is too large to fit
into memory. In this case one can use the \texttt{ff} package to store R
data objects on file, and use them as if they are available in memory.
\texttt{DatabaseConnector} can download data directly into ffdf objects:

\begin{Shaded}
\begin{Highlighting}[]
\NormalTok{x <-}\StringTok{ }\KeywordTok{querySql.ffdf}\NormalTok{(conn, }\StringTok{"SELECT * FROM person"}\NormalTok{)}
\end{Highlighting}
\end{Shaded}

Where x is now an ffdf object.

\hypertarget{querying-different-platforms-using-the-same-sql}{%
\subsection{Querying different platforms using the same
SQL}\label{querying-different-platforms-using-the-same-sql}}

One challenge when writing code that is intended to run on multiple
database platforms is that each platform has its own unique SQL dialect.
To tackle this problem the
\href{https://ohdsi.github.io/SqlRender/}{SqlRender package} was
developed. SqlRender can translate SQL from a single starting dialect
(SQL Server SQL) into any of the platforms supported by
DatabaseConnector. The following convenience functions are available
that first call the \texttt{render} and \texttt{translate} functions in
\texttt{SqlRender}: \texttt{renderTranslateExecuteSql},
\texttt{renderTranslateQuerySql}, \texttt{renderTranslateQuerySql.ffdf}.
For example:

\begin{Shaded}
\begin{Highlighting}[]
\NormalTok{persons <-}\StringTok{ }\KeywordTok{renderTranslatequerySql}\NormalTok{(conn, }
                                   \DataTypeTok{sql =} \StringTok{"SELECT TOP 10 * FROM @schema.person"}\NormalTok{,}
                                   \DataTypeTok{schema =} \StringTok{"cdm_synpuf"}\NormalTok{)}
\end{Highlighting}
\end{Shaded}

Note that the SQL Server-specific `TOP 10' syntax will be translated to
for example `LIMIT 10' on PostgreSQL, and that the SQL parameter
\texttt{@schema} will be instantiated with the provided value
`cdm\_synpuf'.

\hypertarget{inserting-tables}{%
\section{Inserting tables}\label{inserting-tables}}

Although it is also possible to insert data in the database by sending
SQL statements using the \texttt{executeSql} function, it is often
convenient and faster to use the \texttt{insertTable} function:

\begin{Shaded}
\begin{Highlighting}[]
\KeywordTok{data}\NormalTok{(mtcars)}
\KeywordTok{insertTable}\NormalTok{(conn, }\StringTok{"mtcars"}\NormalTok{, mtcars, }\DataTypeTok{createTable =} \OtherTok{TRUE}\NormalTok{)}
\end{Highlighting}
\end{Shaded}

In this example, we're uploading the mtcars data frame to a table called
`mtcars' on the server, that will be automatically created.

\hypertarget{dbi-interface}{%
\section{DBI interface}\label{dbi-interface}}

\texttt{DatabaseConnector} implements the DBI interface for
compatibility with other R packages. One can use the DBI functions
instead of the ones described before, for example:

\begin{Shaded}
\begin{Highlighting}[]
\NormalTok{conn <-}\StringTok{ }\KeywordTok{dbConnect}\NormalTok{(}\KeywordTok{DatabaseConnectorDriver}\NormalTok{(), }
                  \DataTypeTok{dbms =} \StringTok{"postgresql"}\NormalTok{,}
                  \DataTypeTok{server =} \StringTok{"localhost/postgres"}\NormalTok{,}
                  \DataTypeTok{user =} \StringTok{"joe"}\NormalTok{,}
                  \DataTypeTok{password =} \StringTok{"secret"}\NormalTok{,}
                  \DataTypeTok{schema =} \StringTok{"cdm"}\NormalTok{)}
\end{Highlighting}
\end{Shaded}

\begin{verbatim}
#> Connecting using PostgreSQL driver
\end{verbatim}

\begin{Shaded}
\begin{Highlighting}[]
\KeywordTok{dbIsValid}\NormalTok{(conn)}
\end{Highlighting}
\end{Shaded}

\begin{verbatim}
#> [1] TRUE
\end{verbatim}

\begin{Shaded}
\begin{Highlighting}[]
\NormalTok{res <-}\StringTok{ }\KeywordTok{dbSendQuery}\NormalTok{(conn, }\StringTok{"SELECT TOP 3 * FROM person"}\NormalTok{)}
\KeywordTok{dbFetch}\NormalTok{(res)}
\end{Highlighting}
\end{Shaded}

\begin{verbatim}
#>   PERSON_ID GENDER_CONCEPT_ID YEAR_OF_BIRTH
#> 1         1              8507          1975
#> 2         2              8507          1976
#> 3         3              8507          1977
\end{verbatim}

\begin{Shaded}
\begin{Highlighting}[]
\KeywordTok{dbHasCompleted}\NormalTok{(res)}
\end{Highlighting}
\end{Shaded}

\begin{verbatim}
#> [1] TRUE
\end{verbatim}

\begin{Shaded}
\begin{Highlighting}[]
\KeywordTok{dbClearResult}\NormalTok{(res)}
\KeywordTok{dbDisconnect}\NormalTok{(res)}
\end{Highlighting}
\end{Shaded}

\hypertarget{sqlite-support}{%
\section{SQLite support}\label{sqlite-support}}

DatabaseConnector also supports SQLite through the
\href{https://cran.r-project.org/web/packages/RSQLite/index.html}{RSQLite
package}, mainly for testing and demonstration purposes. Provide the
path to the SQLite file as the \texttt{server} argument when connecting.
If no file exists it will be created:

\begin{Shaded}
\begin{Highlighting}[]
\NormalTok{conn <-}\StringTok{ }\KeywordTok{connect}\NormalTok{(}\DataTypeTok{dbms =} \StringTok{"sqlite"}\NormalTok{, }\DataTypeTok{server =} \KeywordTok{tempfile}\NormalTok{())}
\end{Highlighting}
\end{Shaded}

\begin{verbatim}
#> Connecting using SQLite driver
\end{verbatim}

\begin{Shaded}
\begin{Highlighting}[]
\CommentTok{# Upload cars dataset as table:}
\KeywordTok{insertTable}\NormalTok{(conn, }\StringTok{"cars"}\NormalTok{, cars)}

\KeywordTok{querySql}\NormalTok{(conn, }\StringTok{"SELECT COUNT(*) FROM cars;"}\NormalTok{)}
\end{Highlighting}
\end{Shaded}

\begin{verbatim}
#>   COUNT(*)
#> 1       50
\end{verbatim}

\begin{Shaded}
\begin{Highlighting}[]
\KeywordTok{disconnect}\NormalTok{(conn)}
\end{Highlighting}
\end{Shaded}


\end{document}
